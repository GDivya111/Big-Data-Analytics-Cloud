\section{Introduction}
\label{sec:intro}

As the demand for data collection and the need for data analytic tools
and algorithms that operate on large data grows, we find existing solutions
are not flexible or robust enough to satisfy the needs of data analysts. 
To address this growing demand, we designed a data analytics cloud platform
and implemented a data analysis engine. More specifically our platform provides
generic support for data analysis engines of any type and we used this
platform to implement a data analysis engine that crawls the web for
unstructured data, structures and combines data, then visualizes the
structured data. This data analysis engine demonstrates the potential
power and flexibility of our data analytics cloud platform.

Our project considers a user focused on the development of novel and
computationally intensive algorithms that operate on large data sets.
These data sets vary a wide range of industry use cases, such as Insurance
Services data, Financial Services data, Character Recognition data, 
pharmaceutical data, and much more. To develop algorithms that work on data
sets as described above, as well as other data sets, the user needs a
flexible analytic cloud platform offering both Platform as a Service
(PaaS) and Software as a Service (SaaS) packages.

This document serves as a record for the designing, implementing, and
testing of a custom data analytics cloud platform and data analysis engine. 
The rest of this section describes the project at a high level,
including the problem statement (\ref{sec:problem}) and motivation
(\ref{sec:motivation}), as well as issues encountered when designing
custom data analytic platforms (\ref{sec:issues}), and finally, describes
the environment the platform will be implemented on
(\ref{sec:environment}).

The remaining sections of this document go into details about the
requirements, specific details about the system environment, the design and
implementation of the platform, the testing plan and the results of the
tests, and finally the work schedule and distribution of tasks among the
team members. Any additional information that does not fall into these
sections is then placed within an Appendix at the end.

\subsection{Problem Statement}
\label{sec:problem}

Current open-sourced analytic packages are built for a few specific data
sets and data analysts must pick a platform that fits their data best. The
Institute for Next Generation IT Systems (ITng) wishes to provide,
maintain, and use a ``master image'' containing all such services in one
platform. Users will then be able to choose the services they need to fit
their data set and the environment will be set up with minimal
configuration and interaction.

In a data analysis engine, data goes through three major data 
transformation steps. \textit{Data Input}, \textit{In-depth Data Analysis},
and \textit{Presentation of Results}. For each one of these steps there are
many different tools and packages a consumer could use, each with their
own advantages based on the use case and the data set. This project aims
to set up an initial base platform for a data analytics engine, then build 
a data analysis package using this platform. This platform can then be 
expanded further by future developers to support other data analysis
packages.

The final deliverables of this project are an analytics cloud platform built
using OpenStack PackStack on a physical host in the Oscar Lab running Red
Hat Enterprise Linux 7.3 and a data analytics engine with the following
features: web crawling using Apache Nutch, storing unstructured data
with Apache Cassandra, indexing and structuring data with Solr/Lucene,
storing structured data with PostgreSQL, filtering data with Mortar and
visualizing data with Portrait. 

\subsection{Motivation}
\label{sec:motivation}

This project is motivated by the goal to create a data analytics cloud
platform with the following properties:
\begin{enumerate}
  \item \textbf{Flexible Data Analysis, Extendability, and ease of use}

Current open sourced data analytic packages are either too specific to
one data set and not very flexible or they are too generic to
accommodate many data sets and do not operate efficiently by wasting
resources on unneeded packages or software. However, the goal of this
platform is to provide a base for data analytic engines that tailor to a
specific need or data set. Users can then select the specific analysis
engine they need, load it into the analytics platform and get the best
performance by reducing the number of unneeded software and packages.
This base platform will provide generic support for users to define and
create their own data analytic engines and pre-existing data analytic
engines will require minimal configuration in order to operate on a
specific data set.

  \item \textbf{Portability}

Given a base platform, as described in this report, consumers will easily
be able to download an analytics image to their local environment or
configure the base platform in a public cloud and upload images to meet
their computing needs.

  \item \textbf{Robust Structuring of Data} 

This platform has a goal to make discovering and importing unstructured
data into a structured format, which can then be processed by data
analytic packages. The platform should be intuitive, easy with basic configuration and
should expect minimum configuration knowledge by the end user.

\end{enumerate}

\subsection{Issues}
\label{sec:issues}
The design of a data analytics platform to provide a solution to the
problem presented in Section~\ref{sec:problem} must address many issues.
The following list identifies the major issues and briefly discusses
why the issues are important. Details about how the issues are 
addressed by our system are covered in Section~\ref{sec:design}.  

\begin{enumerate}
  \item \textbf{Security}

    When a consumer transitions to the cloud, they effectively give
control of their data to a trusted entity. That trusted entity, the
cloud provider, must be able to make guarantees about the security of their
customers data. For example, who can access or change data, who
has accessed or changed data, how was data changed, etc. Security is a
large task not to be taken lightly and we address some, but not all, of
these security concerns in Section~\ref{sec:security}.

  \item \textbf{Portability}

    Following security, portability of the analysis engines is the
next biggest concern. Due to the high cost of running computationally
expensive algorithms in the cloud, consumers can save money by testing
and developing algorithms locally and then moving to the cloud for
production analysis. Similarly, consumers may find it beneficial
to move from cloud to cloud to capitalize on lower computation prices.
To support this, clouds must have support for similar image formats as
each other and the testing/development environment. The issue of
portability is addressed in Section~\ref{sec:portability}

  \item \textbf{Dealing with Unstructured Data}

    Data collection is happening at rates much faster than it can be
properly stored in a meaningful structured manner, thus it is crucial
the data analysis platform supports an analytics engine that converts
unstructured data into structured data and combines it with existing
structured data. Section~\ref{sec:structuring} discusses the design
decisions taken to accommodate the need for structuring of data.

  \item \textbf{Ease of Use}

    Finally, the entire system should be relatively easy to use.
Although ease of use is subjective and hard to measure, we discuss the
design decisions and argue how the decisions increase the usability of the
platform and data analytics engines in Section~\ref{sec:ease}. In
general, systems are easier to use when presented with intuitive
graphical user interfaces and minimalistic configuration/command line
interaction.

\end{enumerate} 

\section{System Environment}
\label{sec:sys-environment}

This section goes into more detail about the environment introduced in
Section~\ref{sec:environment}. First, it gives the technical
specifications of the hardware environment followed by the
pre-configured software environment and briefly describes the networking
setup and how access to the machine was configured.

\subsection{Hardware}
\label{sec:hardware}

Our data analytics platform is implemented on a server in Venture 3
Suite 153 on North Carolina State University's Centennial Campus. The
server has the following specs:
\begin{itemize}
  \item SuperMicro Server Chassis
  \item 2 6-core Intel Xeon E5-2620 V2 2.10 GHz processors
  \item 128 GB of RAM
  \item 80 GB SSD for the /boot and /home mounts
  \item 500 GB HDD for the swap space and root mount (/)
  \item IBM GPFS file system mounted to an IBM Elastic Storage Server
5146-GL2 via a 10 GbE adapter.
  \item Internet connection via a 40 GbE adapter.
\end{itemize}

\subsection{Software and Network Configuration}
\label{sec:software}

The physical host described above initially comes pre-installed with Red
Hat Enterprise Linux 7.3. VLANs are setup on the Internet connected NIC
with VLAN 26 designated for all Internet traffic and VLAN 23 designated
for storage traffic. In order to connect to the host, users must first
SSH into a proxy setup with a public IP address and further connect to
the host at 10.26.10.10 via a private LAN.


